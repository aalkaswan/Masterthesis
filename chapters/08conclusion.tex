\chapter{Conclusion and Future Work}
\label{conclusion}
\section{Conclusion}
Answer the research questions according to the findings

\section{Future Work}

\begin{itemize}
    \item Exploration of other compiled languages
    \item Evaluation of different decompilers (Radare/Ida/Hopper)
    \item Since Ghidra decompiles some functions badly and generally recovers very few additional details, exploration of the application of these methods to disassembled code instead of decompiled code
    \item Automatic scoring methods to determine the quality of a decompiled sample
    \item The application of semantic scoring methods to this model
    \item Exploration of other, weaker strippers
    \item Developing this model into a Ghidra plugin
    \item Instead of using base Ghidra, use a different model like DIRE to add identifiers to the stripped code
\end{itemize}
\section{Threats to validity}
\subsection{Internal}
\begin{itemize}
    \item Noise, some samples have low-quality comments (Badly written documentation, or incorrectly collected)
    \item Inductive bias from the pre-trained base model
    \item A lucky randomly selected test set
    \item Bias stemming from data collection (only samples that decompile nicely and are commented are represented in the data)
\end{itemize}
\subsection{External} 
\begin{itemize}
    \item This research only covers unstripped and stripped binaries, some code, particularly malware might use methods to resist analysis
    \item We use a very bipolar notion of stripping, some binaries might have been stripped with a significantly weaker stripper
    
\end{itemize}
\subsection{Construct}
\begin{itemize}
    \item We only cover and explore a single state-of-the-art model, other models might be more successful (mono-operation)
    \item Documentation that can be derived from code, might not be useful at all (mono-method) (can be argued about code summarization in general)
\end{itemize}

