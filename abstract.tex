Reverse engineering binaries is required to understand and analyse programs of which the source code is unavailable. Decompilers can transform the largely unreadable binaries into a readable source code-like representation. Yet, during compilation and decompilation many aspects of source code, such as variable names and comments, are lost. Furthermore, by stripping the binaries, all other symbols, like the function name, are removed from the binary.

Reverse engineering is a time-consuming process, much of which is taken up by labeling the functions with semantic information \cite{reverseEngineerProcess}. We therefore propose a novel code summarization method applied to decompiled and stripped decompiled code. We leverage the existing BinSwarm dataset \cite{InlinedFunc}, which is further extended with aligned source code summaries. We create an artificial demi-stripped dataset, by removing the identifiers from unstripped decompiled code. We also fine-tune a pre-trained CodeT5 model for the code summarization task on the given dataset. Furthermore, we investigate the performance of the input types, the impact of data duplication and the importance of each aspect present in source code on the model performance. Finally, we design and present some intermediate-training objectives to increase the model performance.

We present the following findings: Firstly, we find that the model produces good summaries for decompiled code, with similar performance to source C code. The quality of the demi-stripped model was significantly lower but still usable. Stripped performed worse, and produced mostly incorrect and unusable summaries. Secondly, we find that deduplication greatly reduces the performance of the model, putting the performance of decompiled code roughly in line with other decompiled datasets \cite{CodeT5}. Third, we found that the loss of identifiers causes a drop in BLEU-4 score of 35\%, with another 25\% decrease attributable to the increase of decompilation faults caused by stripping. Lastly, we present a Deobfuscation intermediate-training objective, which improved the performance of the model by 0.54 and 1.54  BLEU-4 on stripped and demi-stripped code respectively. 